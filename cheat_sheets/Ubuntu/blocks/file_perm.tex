\begin{block}{File permission}%
	\begin{tabular}{p{\bashcolumnsize}p{\desccolumnsize}}
		\bcom{chmod [permissions] [file]}{Changes the permissions of \texttt{[file]} to \texttt{[permissions]}} Permissions can be set for the \textbf{u}ser (owner of the file), the user \textbf{g}roup the file belongs to, \textbf{o}ther groups, or \textbf{a}ll. Possible permission are \textbf{r}ead, \textbf{w}rite, or e\textbf{x}ecute. Use the following syntax to set permissions: \mbox{\textbf{(ugoa)=(w-)(r-)(x-)}}. (\textbf{-} means the permission is not given.) \\
		\bcom{chown [owner] [file]}{Changes the owner of \texttt{[file]} to \texttt{[owner]}}\\
		\bcom{chgrp [group] [file]}{Changes the user group of \texttt{[file]} to \texttt{[group]}}\\
	\end{tabular}
\end{block}
