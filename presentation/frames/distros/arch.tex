\begin{frame}
	\frametitle{Arch Linux and Manjaro}
	\subsection{Arch Linux and Manjaro}
	
	\begin{columns}
		\begin{column}[t]{.45\linewidth}
			\begin{block}{Arch Linux}
				\begin{itemize}
					\item Around since 2002
					\item Release cycle:
						\begin{itemize}
							\item Rolling release model
							\item Latest stable versions of most software
							\item \textbf{stable} – testing – unstable
							\item Quick access to new versions of software
						\end{itemize}
					\item Package manager: \texttt{pacman}
					\item Additional package repository called \textbf{A}rch \textbf{U}ser \textbf{R}epository
					\item Get the package manager \texttt{yay} if you can
					\item No visual installation!
				\end{itemize}
			\end{block}
		\end{column}
		\hfill
		\begin{column}[t]{.45\linewidth}
			\begin{block}{Manjaro}
				\begin{itemize}
					\item Around since 2011
					\item Derivative of Arch Linux
					\item Default desktop: Xfce / KDE Plasma / GNOME / Phosh
					\item Focus on user-friendliness and accessibility
					\item Still most of the benefits of Arch
					\item Visual installer
					\item Uses the same package repositories as Arch
					\item Package manager: \texttt{pacman}
				\end{itemize}
			\end{block}
		\end{column}
	\end{columns}
	
	\vfill
	
	\begin{columns}
		\begin{column}{.5\linewidth}
			\centering\includegraphics[height=3\baselineskip]{../graphics/logos/Arch/archlinux-logo-dark-1200dpi.b42bd35d5916.png}
		\end{column}
		\hfill
		\begin{column}{.5\linewidth}
			\centering\includegraphics[height=2\baselineskip]{../graphics/logos/Manjaro/Main_page_logo.png}
		\end{column}
	\end{columns}
\end{frame}
