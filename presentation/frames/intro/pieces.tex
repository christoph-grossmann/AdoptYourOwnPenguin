\begin{frame}
	\frametitle{The pieces of a Linux operating system}
	\subsection{The pieces of a Linux operating system}
	
	\begin{enumerate}
		\item \textbf{Bootloader}:\\ The software managing the boot process of your computer and thus responsible for starting the operating system.
		\item \textbf{Kernel}:\\ The actual Linux. It manages the CPU, the memory and all peripheral devices.
		\item \textbf{Init system}:\\ Bootstraps the user space and controls the daemons. Takes over after the bootloader is done.
		\item \textbf{Daemons}:\\ Background services like printing, sound, or scheduling.
		\item \textbf{Graphical Server}:\\ Renders and displays all graphics.
		\item \textbf{Desktop environment}:\\ The piece the user interacts with. There are a few different environments to choose from (GNOME, KDE, Mate, XFCE, etc.).
		\item \textbf{Applications}:\\ Software that is not included with any of the previous pieces. Can be installed via a package manager or by hand.
	\end{enumerate}
\end{frame}
